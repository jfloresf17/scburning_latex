\section{Diseño de investigación}
\subsection{Tipo de investigación}
Es \textbf{básica}, ya que se busca producir nuevo conocimiento con respecto a la identificación y estimación precisa de las áreas 
de quema de caña de azúcar en la costa norte y centro del Perú mediante las imágenes satelitales.

\subsection{Nivel de investigación}
La investigación es de nivel \textbf{descriptiva} porque se busca especificar las características, propiedades y comportamiento 
de la quema de caña azúcar mediante imágenes satelitales y en base a ello, determinar la ubicación y extensión de estos
eventos en la costa norte y centro del Perú. 

\subsection{Enfoque de investigación}
El enfoque es \textbf{cuantitativo}, ya que que utiliza herramientas de análisis matemático y estadístico como parte de los
algoritmos empleados para la detección de la quema de caña de azúcar en las imágenes satelitales.

\subsection{Diseño de investigación}
La investigación es de tipo \textbf{no experimental y longitudinal}.

El tipo de investigación adoptado es no experimental, dado que se emplearon imágenes satelitales 
para la detección de la quema de caña de azúcar. No se llevó a cabo ninguna manipulación 
de las variable, siendo preprocesadas las imágenes únicamente para generar las etiquetas correspondientes. 

Además, se clasifica como un diseño longitudinal, puesto que se recolectaron imágenes satelitales de la costa norte 
y centro del Perú a lo largo del periodo comprendido entre 2017 y 2022.
