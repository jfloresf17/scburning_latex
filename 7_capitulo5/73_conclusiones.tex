\section{Conclusiones y Recomendaciones}
\subsection{Conclusiones}

Este estudio demuestra que es posible desarrollar un modelo de aprendizaje ensamblado para la detección 
de quema de caña de azúcar. Al combinar el aprendizaje profundo con métodos de clasificación por árboles de decisión, 
el modelo ensamblado fue capaz de detectar y delimitar eficazmente las áreas quemadas alcanzando una efectividad 
del 100 \% en la identificación de las áreas quemadas correspondiente a los casos de validación reportados por el OEFA, 
utilizando un umbral óptimo de $0.20$.

El conjunto de datos necesario para el funcionamiento del modelo de aprendizaje ensamblado destinado a 
detectar la quema de caña de azúcar, está consituido por 1,054 imágenes satelitales Sentinel-2, junto con índices de vegetación 
como el NDVI y NDWI, y variables topográficas como la pendiente y la distancia a coberturas agrícolas. Además, se incluyeron 
etiquetas de áreas quemadas, generadas mediante una metodología específica detallada que combina la detección temporal de quema 
basada en reglas y la validación visual de las imágenes detectadas. 

Los algoritmos que mejor se ensamblan al modelo para detectar la quema de caña de azúcar son los de aprendizaje 
profundo y de clasificación por píxeles. En particular, U-Net, un modelo de deep learning que captura bien las relaciones no lineales 
en las imágenes, y LightGBM, un potente algoritmo de aprendizaje automático basado en árboles de decisión, han demostrado ser complementarios en la detección 
de áreas quemadas.

Como resultado, el modelo de aprendizaje ensamblado presenta un desempeño superior en comparación con los algoritmos individuales según las métricas de evaluación: 
82.8 \% en IoU, 86.4 \% en precisión, 95.3 \% en recall, 90.6 \% en F1-Score y 87.5 \% en coeficiente Kappa. 

El modelo de Stacking supera consistentemente a U-Net y LightGBM en cada una de estas 
métricas. Además, los resultados visuales muestran una delimitación más equilibrada y coherente de las áreas quemadas debido a capacidad de combinar las fortalezas de ambos enfoques, heredando la 
coherencia espacial de U-Net y la potente capacidad de clasificación a nivel de píxel de LightGBM.

\subsection{Recomendaciones}

\begin{enumerate}
    \item \textbf{Realizar estudios específicos sobre el ciclo de cultivo de la caña de azúcar}: Implementar investigaciones detalladas utilizando imágenes satelitales para monitorear todas las etapas del proceso de cultivo de caña de azúcar, desde la preparación del terreno hasta la cosecha, para identificar 
    y gestionar mejor las quemas.

    \item \textbf{Incluir detalles en los reportes de emergencia y de recuperación de campos}: Incorporar en los reportes de emergencia datos específicos sobre el área quemada estimada, junto con información detallada proporcionada por los administrados, como cronogramas de fechas planificadas para las quemas, 
    la extensión estimada de las áreas afectadas por las quemas una vez realizadas, y el tiempo estimado de recuperación de las tierras hasta que estén listas para una nueva siembra. Esta información adicional permitirá una validación más precisa y temporalmente coherente de los modelos mediante el uso de imágenes 
    satelitales, mejorando así la capacidad de monitoreo y gestión de las áreas quemadas.

    \item \textbf{Aumentar la base datos con nuevas imágenes en diferentes condiciones}: Ampliar la base de datos con imágenes de satélite en diferentes condiciones climáticas, de iluminación y diversas con presencia de superficies similares a la quema en términos de reflectancia, para mejorar la generalización 
    del modelo y su capacidad para detectar áreas quemadas en diversas situaciones. Asimismo, se recomienda explorar otros métodos de etiquetado, incorporando la información de los reportes de emergencia y de recuperación de campos mencionados en el punto anterior, con el fin de incrementar la cantidad y la calidad 
    de los datos disponibles.

    \item \textbf{Actualización de los Instrumentos de Gestión Ambiental}: Revisar y actualizar los Instrumentos de Gestión Ambiental (IGA) de las empresas azucareras, incluyendo los Planes de Adecuación y Manejo Ambiental (PAMA) para empresas en operación y los Estudios de Impacto Ambiental (EIA) para la ampliación, 
    modificación o formalización de sus operaciones. Esta actualización debe incorporar la implementación de la cosecha mecanizada en verde como una alternativa viable a la quema de caña de azúcar. 
\end{enumerate}


