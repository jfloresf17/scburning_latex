\section{Introducción}
En la costa norte y centro del Perú, la práctica de quemar los residuos de la caña de azúcar ha generado una creciente preocupación debido a sus impactos negativos en el suelo, el medio ambiente y la salud 
de la población circundante. Esta tradicional práctica agrícola libera grandes cantidades de gases de efecto invernadero y partículas contaminantes, deteriorando la calidad del aire y afectando la salud de 
las personas que viven en las zonas cercanas a estos campos de cultivo.

A pesar de los esfuerzos por implementar alternativas más sostenibles, como la cosecha en verde, esta problemática enfrenta desafíos significativos que incluyen la falta de una legislación 
estricta que prohíba la quema de caña y la insuficiente información proporcionada por las empresas azucareras, lo que dificulta el control y la gestión efectiva de esta práctica. La combinación de estos factores 
ha generado insatisfacción entre la población, que exige soluciones más eficaces y sostenibles.

En este contexto, se propone desarrollar un modelo de aprendizaje ensamblado que incorpore técnicas avanzadas de aprendizaje automático, específicamente los árboles de decisión potenciados por gradiente y redes neuronales convolucionales para detectar 
y mapear las áreas con quema de caña de azúcar. 

El objetivo de este modelo es proporcionar información precisa y actualizada sobre la extensión y ubicación de las áreas de quema, facilitando una toma de decisiones más informada por parte de autoridades y gestores ambientales. 
La detección precisa de estas áreas permitirá a las autoridades implementar políticas públicas más efectivas y medidas de mitigación específica para estas áreas, promoviendo prácticas agrícolas más sostenibles y amigables con el medio ambiente.

Se espera que este modelo de detección contribuya significativamente a la mitigación de la quema de caña de azúcar en Perú, promoviendo el desarrollo de soluciones que sean ambientalmente sostenibles y beneficiosas para 
la salud pública. Además, de poder ser utilizado para otros países que enfrentan desafíos similares, demostrando el potencial de la inteligencia artificial en la resolución de problemas ambientales críticos.


