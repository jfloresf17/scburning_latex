\section{Introducción}
En la costa norte y centro del Perú, la práctica de quemar los residuos de la caña de azúcar ha generado una creciente preocupación debido a sus impactos negativos en el suelo, el medio ambiente y la salud 
de la población circundante. Esta tradicional práctica agrícola libera grandes cantidades de gases de efecto invernadero y partículas contaminantes, deteriorando la calidad del aire y afectando la salud de 
las personas que viven en las zonas cercanas a estos campos de cultivo.

A pesar de los esfuerzos por implementar alternativas más sostenibles, como la cosecha en verde, esta problemática enfrenta desafíos significativos que incluyen la falta de una legislación 
estricta que prohíba la quema de caña y la insuficiente información proporcionada por las empresas azucareras, lo que dificulta el control y la gestión efectiva de esta práctica. La combinación de estos factores 
ha generado insatisfacción entre la población, que exige soluciones más eficaces y sostenibles.

En este contexto, se propone desarrollar un modelo de aprendizaje ensamblado que incorpore técnicas avanzadas de aprendizaje automático, específicamente los árboles de decisión potenciados por gradiente y redes 
neuronales convolucionales para detectar y mapear las áreas con quema de caña de azúcar. 

El objetivo de este modelo es proveer información precisa y actualizada sobre la extensión y ubicación de estas zonas, respaldando así una toma de decisiones más informada a fin de mejorar la fiscalización
de esta práctica agrícola de las empresas azucareras por parte de los organismos competentes.

Se espera que el modelo facilite la identificación oportuna de dichas áreas, dirigiendo los esfuerzos de monitoreo hacia las unidades fiscalizables que lo requieran. De este modo, sería posible redefinir y modificar 
los instrumentos de gestión ambiental para prevenir y controlar la quema de caña de azúcar, incorporando métodos de cosecha más sostenibles. Además, su aplicación podría extenderse a otros países con desafíos similares, 
demostrando el potencial de la inteligencia artificial para resolver problemáticas como la detección de áreas quemadas.

