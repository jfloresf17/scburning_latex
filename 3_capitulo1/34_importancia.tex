\section{Importancia y alcance de la investigación}

\subsection{Importancia}
El modelo desarrollado en este estudio será fundamental para la detección precisa de áreas con quema de caña de azúcar, 
permitiendo al Organismo de Evaluación y Fiscalización Ambiental (OEFA) identificar y validar los predios agrícolas y las empresas 
responsables de estas prácticas. Esta herramienta mejorará la capacidad de fiscalización ambiental, facilitando la toma de medidas correctivas 
contra los infractores y promoviendo el cumplimiento de las normativas ambientales. 

Además, el modelo permite un seguimiento casi en tiempo real de las quemas, ya que puede procesar nuevas imágenes tan pronto como se actualicen los 
productos satelitales. También será útil para analizar eventos pasados donde se tiene conocimiento de quemas reportadas, pero cuya extensión no ha 
sido estimada. Esto permitirá un análisis retroactivo de los impactos, mejorando la planificación de estrategias de mitigación.

\subsection{Alcance}
El modelo desarrollado en este estudio se aplica específicamente a las regiones de la costa norte y centro del Perú, utilizando una base de datos 
y los modelos preentrenados para estas áreas, disponibles públicamente en \href{https://huggingface.co/datasets/jfloresf/scburning}{Hugging Face}. 

Este enfoque regional permite una adaptación precisa a las condiciones locales, pero el modelo también tiene un gran potencial de escalabilidad, es por ello que el código fuente 
está disponible en \href{https://github.com/jfloresf17/scburning}{GitHub} para su replicación y futura aplicación en otros contextos geográficos y tipos de cultivos, lo que 
permite su uso más allá de las zonas estudiadas. 

Este estudio también sienta las bases para futuras investigaciones en la detección de quemas y en la aplicación de 
tecnologías de aprendizaje automático en la agricultura y el monitoreo ambiental, ofreciendo un marco flexible y adaptable para diversas necesidades de gestión y 
sostenibilidad.