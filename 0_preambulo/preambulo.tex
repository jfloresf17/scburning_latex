% Configuración de la bibliografía -------------------------------------
%% Citación en estilo APA
\usepackage[spanish]{babel}
\usepackage[utf8]{inputenc}
\usepackage[natbibapa]{apacite}
%%\usepackage[natbibapa]{apacite}
% \AtBeginDocument{%
%   \renewcommand{\BOthers}[1]{et al.}
%   \renewcommand{\BBAB}{\&}
% }

\bibliographystyle{apalike-es}
%% Espaciamiento en la bibliografía
\setlength{\bibsep}{5pt plus 0.3ex}

% ---------------------------------------------------------------------- 
% Establecer formato de la fuente --------------------------------------

%% Fuente utilizado por defecto ----------------------------------------
\usepackage[T1]{fontenc}

%% Fuente Times New Roman no compatible con pdfLaTeX -------------------  
\usepackage{fontspec}
\setmainfont{Times New Roman}

%% Permite el uso de @ en los nombres de los comandos ------------------
\makeatletter

% ----------------------------------------------------------------------

% Cambia coma por punto a los números ----------------------------------
\decimalpoint  

% ----------------------------------------------------------------------

% Utilizar comandos para las ecuaciones --------------------------------
\usepackage{amsmath}
% ----------------------------------------------------------------------

% Establecer formato para las tablas -----------------------------------

%% Para definir tablas con ancho fijo ----------------------------------
\usepackage{tabularx}

%% Para definir tablas con columnas anchas -----------------------------
\usepackage{longtable}

%% Crea celdas que abarcan varias filas en una tabla -------------------
\usepackage{multirow} 
% ----------------------------------------------------------------------

% Establecer márgenes --------------------------------------------------
%% Márgenes de la página igual para todos los lados --------------------
\usepackage[margin=2.54cm]{geometry} 

% ----------------------------------------------------------------------


% Establecer encabezado y pie de página --------------------------------

%% Borra todo el encabezado --------------------------------------------
\usepackage{fancyhdr} 
\pagestyle{fancy}
\fancyhf{}

%% Elimina la línea horizontal del encabezado --------------------------
\renewcommand{\headrulewidth}{0pt} 

%% Muestra el número de página en el centro del pie de página ----------
\fancyfoot[C]{\thepage} 

% ----------------------------------------------------------------------

% Establecer formatos dentro del contenido -----------------------------

%% Circulos para viñetas (Lista) ---------------------------------------
\renewcommand{\labelitemi}{$\bullet$} 

%% Para los titulos de las secciones -----------------------------------
\usepackage{titlesec} 

%% Para los titulos de las índices -------------------------------------
\usepackage{tocloft} %Titulos de ÍNDICES

%% Crea enlaces de color negro en el documento -------------------------
\usepackage[pdfencoding=auto, colorlinks=true,linkcolor=negro,
            citecolor = negro]{hyperref}

%% Proporciona soporte para caracteres matemáticos Unicode -------------
\usepackage[mathbf=sym]{unicode-math} 

%% Aplicar SANGRÍA -----------------------------------------------------
\setlength{\parindent}{1cm}

%% Aplicar espaciado 2cm -----------------------------------------------
\usepackage{setspace}
\doublespacing

% ----------------------------------------------------------------------


% Crear estilo para los códigos ----------------------------------------

%% Definir los colores -------------------------------------------------
\usepackage{xcolor} % Add this line to define the color
\definecolor{granate}{RGB}{113,22,16}
\definecolor{gris}{RGB}{154,153,157}
\definecolor{arena}{RGB}{230,217,170}
\definecolor{azul}{rgb}{0.03, 0.15, 0.4}
\definecolor{negro}{rgb}{0, 0, 0}
\definecolor{purpura}{rgb}{0.2,0,1}
\definecolor{rojo}{rgb}{0.7,0,0.3}
\definecolor{verde}{rgb}{0,0.6,0}

%% Definición de estilo ------------------------------------------------ 
\usepackage{listings}
\lstdefinestyle{mystyle}{
  backgroundcolor=\color{negro},  
  commentstyle=\color{verde},
  keywordstyle=\color{rojo},
  numberstyle=\tiny\color{gris},
  stringstyle=\color{purpura},
  basicstyle=\footnotesize,
  breakatwhitespace=false,         
  breaklines=true,                 
  captionpos=b,                    
  keepspaces=true,                 
  numbers=left,                    
  numbersep=5pt,                  
  showspaces=false,                
  showstringspaces=false,
  showtabs=false,                  
  tabsize=2
}
\lstset{style=mystyle}

% ----------------------------------------------------------------------


% Definir titulos de las secciones -------------------------------------

%% Formato de CAPÍTULO (NIVEL 1) ---------------------------------------
\titleformat{\chapter}[block]{\normalfont\normalsize\bfseries\centering}
{CAPÍTULO \thechapter \\}{0.5em}{\normalsize}
\titlespacing*{\chapter}{0pt}{-12 pt}{5 pt}

%% Formato de section (NIVEL 2) ----------------------------------------
\titleformat{\section}[block]{\normalfont\normalsize\bfseries}
{\thesection.}{0.5em}{\normalsize}
\titlespacing*{\section}{0pt}{12 pt}{5 pt}

%% Formato de subsection (NIVEL 3) -------------------------------------
\titleformat{\subsection}[block]{\normalfont\normalsize\bfseries
\itshape}{\thesubsection.}{0.5em}{\normalsize}
\titlespacing*{\subsection}{0pt}{12 pt}{5 pt}

%% Formato subsubsection (NIVEL 4) -------------------------------------
\titleformat{\subsubsection}[runin]{\normalfont\normalsize\bfseries}
{\thesubsubsection.}{0.5em}{\normalsize}
\titlespacing*{\subsubsection}{\parindent}{12 pt}{5 pt}

%% Crear subsubsubsection (NIVEL 5) ------------------------------------
\titleformat{\paragraph}[runin]{\normalfont\normalsize\bfseries
\itshape}{\theparagraph.}{0.5em}{\normalsize}
\titlespacing*{\paragraph}{\parindent}{12 pt}{5 pt}

%% Mostrar los niveles en el índice ------------------------------
\setcounter{tocdepth}{5}
\setcounter{secnumdepth}{5}

% ----------------------------------------------------------------------

% Modificar estilo de figuras, tablas y ecuaciones ---------------------

%% Evitar reiniciar el contador en cada capítulo -----------------------
\usepackage{chngcntr}
\counterwithout{figure}{chapter}  
\counterwithout{table}{chapter}  
\counterwithout{equation}{chapter}

%% Configurar el formato de las figuras para centrar el nombre ---------
\usepackage{float} %% Para usar [H] en las figuras
\usepackage{graphicx} %% Para usar \includegraphics
\usepackage{caption} %% Para usar \caption
\usepackage{booktabs} %% Para usar \toprule, \midrule, \bottomrule
\usepackage{bookmark}
\captionsetup[figure]{
    position=top, % Etiqueta de la tabla abajo
    labelformat=simple, % Elimina la etiqueta Figura
    labelsep=newline, % Salto de línea después de la etiqueta
    font={normalsize, it}, % Tamaño de la fuente
    singlelinecheck=false % Alineación del nombre en caso de que ocupe 
                          % más de una línea
}

%% Configurar el formato de las tablas para centrar el nombre ----------
\captionsetup[table]{
    position=top, % Etiqueta de la tabla abajo
    labelformat=simple, % Elimina la etiqueta Tabla
    labelsep=newline, % Salto de línea después de la etiqueta
    font={normalsize, it}, % Tamaño de la fuente
    singlelinecheck=false % Alineación del nombre en caso de que ocupe 
                          % más de una línea
}

%% Cambiando las etiquetas de las FIGURAS y TABLAS ---------------------
\addto\captionsspanish{\renewcommand{\figurename}{\normalfont\bfseries 
Figura}} 
\addto\extrasspanish{\def\figureautorefname{Figura}}
 
\addto\captionsspanish{\renewcommand{\tablename}{\normalfont\bfseries 
Tabla}}
\addto\extrasspanish{\def\tableautorefname{Tabla}} 

% ----------------------------------------------------------------------


% Cambiando a Números Romanos solo los Capítulos ----------------------------
\renewcommand{\thechapter}{\bfseries{\Roman{chapter}}}

% Cambiando a Números Arábigos las Secciones, Ecuaciones, etc. ---------
\renewcommand{\thesection}{\arabic{chapter}.\arabic{section}}
\renewcommand{\theequation}{\arabic{equation}}
\renewcommand{\thetable}{\arabic{table}}
\renewcommand{\thefigure}{\arabic{figure}}

% ----------------------------------------------------------------------


% Escribir los INFORMACIÓN PERSONAL Y DEL TRABAJO ----------------------

%% Autor para CARÁTULA (Siempre en mayuscula y sin saltos de linea) ----
\authorcaratula{Jair Francisco FLORES FARFAN}

%% Asesor -------------------------------------------------------------
\asesor{Dr. Francisco Alejandro ALCÁNTARA BOZA}

%% Título para CARÁTULA (Siempre en mayuscula y sin saltos de linea) ---
\titlecaratula{Desarrollo de un modelo de aprendizaje ensamblado para la detección
de quema de caña de azúcar en la costa norte y centro del Perú durante el período 2017 - 2022.}

%% Nombre de la FACULTAD -----------------------------------------------
\facultad{Facultad de Ingeniería Geológica, Minera, Metalúrgica y 
Geográfica}

%% Nombre de la ESCUELA ------------------------------------------------
\eap{Escuela Académico Profesional de Ingeniería Geográfica}

%% Para obtener el título profesional de -------------------------------
\grado{Ingeniero Geógrafo}

%% AÑO para CARÁTULA ---------------------------------------------------
\yyearr{2024}

% ----------------------------------------------------------------------
