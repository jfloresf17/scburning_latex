\section{Operacionalización de variables}
\begin{table}[H]
    \centering
    \caption{Cuadro de Operacionalización de variables}
    \label{tab:operacionalizacion}
    \begin{tabular}{m{3cm}m{3cm}m{2.5cm}m{2cm}m{3.9cm}}
        \hline
        \textbf{Variables} & \textbf{Dimensión} & \textbf{Indicador} & \textbf{Unidad de medida} & \textbf{Instrumento} \\
        \hline
        \multicolumn{5}{c}{\textbf{Variable Independiente}} \\
        \hline
        \multirow{6.25}{3.5cm}{Desarrollo de un modelo de aprendizaje ensamblado.} & Modelo U-Net & Métricas de evaluación & Porcentaje & Librería Torch y PyTorchLightning \\
        & Gradient Boosting Decision Tree & Métricas de evaluación & Porcentaje & Librería LightGBM en Python \\
        & Regresión Logística & Métricas de evaluación & Porcentaje & Librería scikit-learn en Python \\
        \hline
        \multicolumn{5}{c}{\textbf{Variable Dependiente}} \\
        \hline
        \multirow{4}{3.5cm}{Detección de quema de caña de azúcar.} & Ubicación de eventos de quema & Coordenadas geográficas & Grados decimales & Plataforma SIG (QGIS) \\
        & Extensión de las áreas quemadas & Área quemada & Hectáreas & Plataforma de análisis de imágenes satelitales (R y Python) \\
        \hline
    \end{tabular}
    \begin{flushleft}
        \textit{Nota.} Elaboración propia.
        \vspace{-\baselineskip}
    \end{flushleft}
\end{table}


